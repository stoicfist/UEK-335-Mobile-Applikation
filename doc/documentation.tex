\documentclass{article}
\usepackage{babel, ngerman}
\usepackage{graphicx}
\usepackage{float}

\begin{document}

% ===================== TITELBLATT =====================
\begin{titlepage}
  \centering
  
  {\Large \textbf{Modul M335 – Mobile Applikationen realisieren}}\\[6pt]
  {\large Dokumentation}\\[18pt]

  \rule{\textwidth}{0.4pt}\\[14pt]
  
  {\huge \textbf{MotoTrack}}\\[6pt]
  {\large Ein mobiler Motorrad-Routen-Tracker}\\[18pt]

  \rule{\textwidth}{0.4pt}\\[30pt]

  \begin{tabular}{@{}ll}
    \textbf{Dozent:} & Urs Beyeler \\   
    \textbf{Autor:} & Peter Ngo \\
    \textbf{Klasse:} & 23-F \\ % falls du eine Klasse brauchst, sonst löschen
    \textbf{Datum:} & 8. Dezember 2025 \\
  \end{tabular}

  \vfill

  {\small Diese Dokumentation beschreibt Idee, Anforderungen, Datenmodell,
  Wireframes und technische Umsetzung der MotoTrack-App.}
\end{titlepage}

\subsection*{1. Projektidee konkretisieren}

MotoTrack ist eine Mobile Hybrid-App, welche Motorradfahrern ermöglicht, ihre Touren per GPS aufzuzeichnen und auf einer Karte darzustellen. Die App konzentriert sich auf grundlegende Funktionen, die für eine einfache Routenerfassung notwendig sind: aktuelle Position anzeigen, Fahrtrichtung darstellen, Route mitschneiden und Tourdaten speichern.

Die Anwendung besteht aus vier Ansichten:
\begin{itemize}
    \item \textbf{Navigation / Karte} – Anzeige von Position und Fahrtrichtung
    \item \textbf{Live-Tracking} – Aufzeichnung der Strecke als Linie
    \item \textbf{Routen-Statistik} – Distanz, Dauer, Geschwindigkeit
    \item \textbf{Einstellungen} – Dark Mode und App-Informationen
\end{itemize}

Die App wird mit Ionic, Angular und Capacitor entwickelt und soll sowohl als PWA wie auch als Android-App lauffähig sein.

\subsection*{2. User Stories}

\textbf{Navigation / Karte}
\begin{itemize}
    \item Als Motorradfahrer möchte ich meine aktuelle Position auf einer Karte sehen, um mich orientieren zu können.
    \item Als Motorradfahrer möchte ich die Fahrtrichtung sehen, um meine Ausrichtung besser einschätzen zu können.
\end{itemize}

\textbf{Live-Tracking}
\begin{itemize}
    \item Als Motorradfahrer möchte ich das Tracking starten und stoppen können, damit meine Route aufgezeichnet wird.
    \item Als Motorradfahrer möchte ich die gefahrene Strecke als Linie sehen, um den Routenverlauf zu erkennen.
\end{itemize}

\textbf{Routen-Statistik}
\begin{itemize}
    \item Als Nutzer möchte ich Distanz, Dauer und Durchschnittsgeschwindigkeit sehen, um die Tour auswerten zu können.
    \item Als Nutzer möchte ich die Tour als JSON exportieren können, um sie weiterzuverwenden.
\end{itemize}

\textbf{Einstellungen}
\begin{itemize}
    \item Als Nutzer möchte ich einen Dark Mode aktivieren können.
    \item Als Nutzer möchte ich gespeicherte Daten löschen können.
\end{itemize}

\subsection*{3. Planung}

\subsubsection*{3.1 Datenmodell (Supabase)}

Das Datenmodell der Anwendung ist bewusst schlank gehalten und besteht aus einer einzelnen Tabelle \texttt{tours}. 
Sie speichert alle relevanten Daten einer aufgezeichneten Motorradtour, darunter Distanz, Dauer und die vollständige Route als JSON-Struktur.

\begin{itemize}
    \item \texttt{id} – UUID (Primärschlüssel)
    \item \texttt{created\_at} – Timestamp
    \item \texttt{duration} – Integer (Sekunden)
    \item \texttt{distance} – Float (Kilometer)
    \item \texttt{average\_speed} – Float
    \item \texttt{route\_points} – JSON (Liste an GPS-Koordinaten)
\end{itemize}

\begin{figure}[h]
\centering
\includegraphics[width=0.55\textwidth]{./figures/datenbank.png}
\caption{Datenbankmodell der Tabelle \texttt{tours}}
\end{figure}

Die Route wird als JSON gespeichert, da sie eine dynamische Menge von GPS-Punkten enthält und somit flexibel und effizient abgelegt werden kann.
\begin{verbatim}
[
  { "lat": 47.12345, "lng": 8.12345, "timestamp": 1681234567890 }
]
\end{verbatim}

\subsubsection*{3.2 Wireframes (Beschreibung)}

\textbf{Navigation / Karte}
\begin{itemize}
    \item Karte mit aktuellem Standort
    \item Darstellung der Fahrtrichtung
    \item Button „Tracking starten“
\end{itemize}

\textbf{Live-Tracking}
\begin{itemize}
    \item Karte mit aufgezeichneter Route
    \item Anzeige der aktuellen Distanz
    \item Button „Tracking stoppen“
\end{itemize}

\textbf{Routen-Statistik}
\begin{itemize}
    \item Anzeige von Distanz, Dauer und Durchschnittsgeschwindigkeit
    \item Buttons: „Tour speichern“ und „Exportieren“
\end{itemize}

\textbf{Einstellungen}
\begin{itemize}
    \item Dark Mode Toggle
    \item App-Informationen
    \item Button „Daten löschen“
\end{itemize}

Die folgenden Abbildungen zeigen die Benutzeroberflächen der Anwendung in Form
von UI-Mockups. Die Screenshots stammen aus der finalen Umsetzung der App und
entsprechen dem geplanten Layout der einzelnen Ansichten.

\begin{figure}[H]
\centering
\includegraphics[width=0.6\textwidth]{./figures/tab1.png}
\caption{UI-Mockup – Navigation / Aufnahme}
\end{figure}

\begin{figure}[H]
\centering
\includegraphics[width=0.6\textwidth]{./figures/tab2.png}
\caption{UI-Mockup – Meine Routen}
\end{figure}

\begin{figure}[H]
\centering
\includegraphics[width=0.6\textwidth]{./figures/tab3.png}
\caption{UI-Mockup – Routen-Statistik}
\end{figure}

\begin{figure}[H]
\centering
\includegraphics[width=0.6\textwidth]{./figures/tab4.png}
\caption{UI-Mockup – Einstellungen}
\end{figure}


\subsubsection*{3.3 UI-Elemente (Ionic Components)}

\begin{itemize}
    \item \texttt{ion-header}, \texttt{ion-toolbar}, \texttt{ion-title}
    \item \texttt{ion-content}
    \item \texttt{ion-button}
    \item \texttt{ion-icon}
    \item \texttt{ion-list}, \texttt{ion-item}
    \item \texttt{ion-card}
    \item \texttt{ion-toggle}
    \item \texttt{ion-tabs}
\end{itemize}

\subsubsection*{3.4 Offline-Tracking und Synchronisation}

Die Anwendung unterstützt das Aufzeichnen von Touren ohne aktive Internetverbindung. 
Wenn das Gerät offline ist, werden abgeschlossene Touren lokal im \texttt{LocalStorage} 
unter dem Schlüssel \texttt{pendingTours} gespeichert. 

Sobald wieder eine Internetverbindung besteht, versucht die App automatisch, alle 
lokal gespeicherten Touren mit Supabase zu synchronisieren. Nach einem erfolgreichen 
Upload werden die entsprechenden Einträge aus dem lokalen Speicher entfernt.

Dieses Verhalten stellt sicher, dass Touren auch in Regionen ohne Mobilfunkempfang 
aufgezeichnet werden können und später zuverlässig nachgeladen werden.

\subsubsection*{3.5 Entwicklung: OSRM und CORS-Proxy}

Während der lokalen Entwicklung kann es zu CORS-Problemen kommen, wenn Routing-Anfragen 
direkt an die öffentliche OSRM-API (\texttt{router.project-osrm.org}) gestellt werden. 
Um dies zu umgehen, wurde eine Proxy-Konfiguration (\texttt{proxy.conf.json}) eingerichtet.

Der Entwicklungsserver wird mit folgendem Befehl gestartet:

\begin{verbatim}
npm run start:proxy
\end{verbatim}

Dadurch werden Anfragen an \texttt{/osrm/*} korrekt weitergeleitet, 
und die App kann ohne CORS-Fehler getestet werden.

\subsection*{Zusammenfassung}

MotoTrack bietet eine kompakte Lösung zur Aufzeichnung von Motorradtouren. Die App umfasst vier übersichtliche Ansichten, nutzt GPS- und Sensorfunktionen und setzt auf moderne Webtechnologien. Der Funktionsumfang ist bewusst schlank gehalten, damit die Umsetzung innerhalb des ÜK-Moduls realisierbar bleibt.

\end{document}
